%%%%%%%%%%%%%%%%%%%%%%%%%%%%%%%%%%%%%%%%%
% Compact Academic CV
% LaTeX Template
% Version 2.0 (6/7/2019)
%
% This template originates from:
% https://www.LaTeXTemplates.com
%
% Authors:
% Dario Taraborelli (http://nitens.org/taraborelli/home)
% Vel (vel@LaTeXTemplates.com)
%
% License:
% CC BY-NC-SA 3.0 (http://creativecommons.org/licenses/by-nc-sa/3.0/)
%
%%%%%%%%%%%%%%%%%%%%%%%%%%%%%%%%%%%%%%%%%

%----------------------------------------------------------------------------------------
%	PACKAGES AND OTHER DOCUMENT CONFIGURATIONS
%----------------------------------------------------------------------------------------

\documentclass[11pt]{article} % Default document font size

%%%%%%%%%%%%%%%%%%%%%%%%%%%%%%%%%%%%%%%%%
% Compact Academic CV
% Structural Definitions
% Version 1.0 (6/7/2019)
%
% This template originates from:
% https://www.LaTeXTemplates.com
%
% Authors:
% Dario Taraborelli (http://nitens.org/taraborelli/home)
% Vel (vel@LaTeXTemplates.com)
%
% License:
% CC BY-NC-SA 3.0 (http://creativecommons.org/licenses/by-nc-sa/3.0/)
%
%%%%%%%%%%%%%%%%%%%%%%%%%%%%%%%%%%%%%%%%%

%----------------------------------------------------------------------------------------
%	REQUIRED PACKAGES AND MISC CONFIGURATIONS
%----------------------------------------------------------------------------------------

\usepackage{graphicx} % Required for including images

\setlength{\parindent}{0pt} % Stop paragraph indentation

%----------------------------------------------------------------------------------------
%	MARGINS
%----------------------------------------------------------------------------------------

\usepackage{geometry} % Required for adjusting page dimensions and margins

\geometry{
	paper=a4paper, % Paper size, change to letterpaper for US letter size
	top=3.25cm, % Top margin
	bottom=4cm, % Bottom margin
	left=3.5cm, % Left margin
	right=3.5cm, % Right margin
	headheight=0.75cm, % Header height
	footskip=1cm, % Space from the bottom margin to the baseline of the footer
	headsep=0.75cm, % Space from the top margin to the baseline of the header
	%showframe, % Uncomment to show how the type block is set on the page
}

%----------------------------------------------------------------------------------------
%	FONTS
%----------------------------------------------------------------------------------------

\usepackage[utf8]{inputenc} % Required for inputting international characters
\usepackage[T1]{fontenc} % Output font encoding for international characters

\usepackage[semibold]{ebgaramond} % Use the EB Garamond font with a reduced bold weight

%----------------------------------------------------------------------------------------
%	SECTION STYLING
%----------------------------------------------------------------------------------------

\usepackage{sectsty} % Allows changing the font options for sections in a document

\sectionfont{\fontsize{13.5pt}{18pt}\selectfont} % Set font options for sections
\subsectionfont{\mdseries\scshape\normalsize} % Set font options for subsections
\subsubsectionfont{\mdseries\upshape\bfseries\normalsize} % Set font options for subsubsections

%----------------------------------------------------------------------------------------
%	MARGIN YEARS
%----------------------------------------------------------------------------------------

\usepackage{marginnote} % Required to output text in the margin

\newcommand{\years}[1]{\marginnote{\scriptsize #1}} % New command for adding years to the margin
\renewcommand*{\raggedleftmarginnote}{} % Left-align the years in the margin
\setlength{\marginparsep}{-10pt} % Move the margin content closer to the text
\reversemarginpar % Margin text to be output into the left margin instead of the default right margin

%----------------------------------------------------------------------------------------
%	COLOURS
%----------------------------------------------------------------------------------------

\usepackage[usenames, dvipsnames]{xcolor} % Required for specifying colours by name

%----------------------------------------------------------------------------------------
%	LINKS
%----------------------------------------------------------------------------------------

\usepackage[bookmarks, colorlinks, breaklinks]{hyperref} % Required for links

% Set link colours
\hypersetup{
	linkcolor=blue,
	citecolor=blue,
	filecolor=black,
	urlcolor=MidnightBlue
}
 % Include the file specifying the document structure and styling

% Set PDF meta-information
\hypersetup{
	pdftitle={Orlando Eduardo - Curriculum vitae},
	pdfauthor={Orlando Eduardo Mart\'inez Durive}
}

%----------------------------------------------------------------------------------------

\begin{document}

%----------------------------------------------------------------------------------------
%	CONTACT AND GENERAL INFORMATION
%----------------------------------------------------------------------------------------

{\LARGE\bfseries Orlando Eduardo Mart\'inez Durive} % Name
\bigskip\bigskip\medskip % Whitespace

\#4125 entre 41 \& 43 \\ % Address
Calle 146B\\ La Lisa, La Habana 17000 Cuba
\medskip % Whitespace

Tel\'efono:  +537 262-8701\\ % Phone number
M\'ovil: +535 805-3104 % Mobile number
\medskip % Whitespace

Email: \href{mailto:landy@fisica.uh.cu}{landy@fisica.uh.cu}\\ % Email address
%\textsc{url}: \href{http://www.ias.edu/spfeatures/einstein/}{http://www.ias.edu/spfeatures/einstein/}\\ % Academic/personal website

%\vspace{0.06\textheight} % Whitespace between contact information and specific CV information

%------------------------------------------------

Nacimiento: 29 de Julio de 1992---Habana, Cuba\\ % Date of birth
Nacionalidad: Cubano % Nationality

%------------------------------------------------

\section*{Posici\'on actual}

Profesor, Facultad de F\'isica, Universidad de la Habana, Cuba. % Current or most recent employment position

%------------------------------------------------

\section*{Especialidades}

Matem\'aticas, Algoritmos, Optimizaci\'on, Aprendizaje de m\'aquina, Desarrollo web, Bases de datos relacionales, Sistemas de informaci\'on geogr\'aficos. % Primary areas of research interest

%----------------------------------------------------------------------------------------
%	WORK EXPERIENCE
%----------------------------------------------------------------------------------------

\section*{Experiencia Laboral}

\years{2015-2017} Revista Temas - Desarrollo web \\
\years{2017-2018} Instituto Nacional Agroforestal - Desarrollo web \\
\years{2018-2018} Muestra Joven - Desarrollo web \\
\years{2018-2018} Instituto Cubano del Cine - Desarrollo web \\
\years{2017-2020} Facultad de F\'isica, Universidad de la Habana \\

%----------------------------------------------------------------------------------------
%	EDUCATION
%----------------------------------------------------------------------------------------

\section*{Educaci\'on}

\years{2012-2017}\textsc{Lic.} en Ciencia de la Computaci\'on, Universidad de la Habana.\\
\years{2019-2020}\textsc{} Actualmente estudiante de Maestr\'ia en Ciencia de la Computaci\'on (con defensa prevista en Septiembre de 2020), Universidad de la Habana.\\

%----------------------------------------------------------------------------------------
%	GRANTS, HONOURS AND AWARDS
%----------------------------------------------------------------------------------------

%\section*{Premios}

%\years{2017} Gran Premio de la Jornada Cient\'ifica de MATCOM.

%----------------------------------------------------------------------------------------
%	PUBLICATIONS AND TALKS
%----------------------------------------------------------------------------------------

\section*{Conferencias}

\years{2019} Movilidad humana en Cuba, usando datos de tel\'efonia m\'ovil, presentado en la Escuela Internacional del Dengue en el Instituto de Medicina Tropical, Pedro Kour\'i (IPK), Cuba.

\years{2020} Simulaci\'on de epidemias, basado en datos de tel\'efonia m\'ovil. Encuentro Nacional de J\'ovenes F\'isicos, Instituto de Cibern\'etica, F\'isica y Matem\'atica (ICIMAFT), Cuba.

%\subsection*{Journal articles}

%\years{1901}Einstein, Albert (1901), “Folgerungen aus den Capillaritätserscheinungen (Conclusions Drawn from the Phenomena of Capillarity)", \emph{Annalen der Physik} 4: 513\\


%------------------------------------------------

%\subsection*{Newspaper articles}

%\years{1940}Einstein, Albert, et al. (December 4, 1948), “To the editors", %\emph{New York Times}\\
%\years{1949}Einstein, Albert (May 1949), “Why Socialism?", \emph{Monthly Review}.

%----------------------------------------------------------------------------------------
%	TEACHING
%----------------------------------------------------------------------------------------

\section*{Cursos impartidos}

\years{2017} M\'etodos Computacionales I\\
\years{2017 \& 2018} An\'alisis Matem\'atico I\\
\years{2018 \& 2019} M\'etodos N\'umericos\\
\years{2019} Probabilidades y Estad\'istica\\
\years{2019} \'Algebra II \\
\years{2020} M\'etodos Computacionales II

\section*{Proyectos}
\begin{itemize}
	\item Datos de mobilida para los epidemiólogos, basado en telefon\'ia m\'ovil, finaciado por la Organizaci\'on Panamericana de la Salud (OPS/PAHO)\\ \href{https://www.paho.org/ish/index.php/en/is4h-in-guyana/12-is4h-stories/75-cuba-wants-to-use-data-from-mobile-phones-to-control-outbreaks-of-tropical-diseases}{https://www.paho.org/ish/index.php/en/is4h-in-guyana/12-is4h-stories/75-cuba-wants-to-use-data-from-mobile-phones-to-control-outbreaks-of-tropical-diseases}
\end{itemize}

\section*{Patentes}
\begin{itemize}
	\item T\'itulo: DBPhoneFlow. \\ 
	N\'umero de solicitud: 0706-02-2020. \\
	Entidad Titular: Facultad de Física, Universidad de la Habana. \\
	Inventores: Orlando Eduardo Martínez Durive, Roger Casimiro Martínez y Alejandro Lage Castellanos.
\end{itemize}

\section*{Estancias en centros de reconocido prestigio internacional}
\begin{itemize}
	\item {\bf Polit\'ecnico de Tur\'in, Italia}: Estancia de 3 semanas de investigaci\'on en Julio - Agosto de 2019 de Postgrado.
\end{itemize}

\section*{Intereses actuales}
\begin{itemize}
	\item Aprendizaje de m\'aquina
	\item Sistemas de bases de datos no relacionales
	\item Ciberseguridad
	\item Sistemas de informaci\'on geogr\'aficos
	\item An\'alisis y visualizaci\'on de grandes conjuntos de datos
\end{itemize}

%------------------------------------------------

%\section*{Service to the profession}

%\ldots

\vfill % Whitespace before final footer

%----------------------------------------------------------------------------------------
%	FINAL FOOTER
%----------------------------------------------------------------------------------------

% Any final footer text such as a URL to the latest version of this CV, last updated date, compiled in XeTeX, etc
%\begin{center}
%	\scriptsize
%	Last updated: \today
%\end{center}

%----------------------------------------------------------------------------------------

\end{document}
